\documentclass[11pt]{report}
\usepackage{bm}
\usepackage[utf8]{inputenc}
\usepackage[spanish]{babel}
\usepackage{dirtytalk}
\usepackage[none]{hyphenat} %Para evitar el corte de palabras
\usepackage{amsmath}
\usepackage{amsthm} %Para definir ambientes con \newtheorem
\usepackage{amsfonts}
\usepackage{amssymb}
\usepackage{makeidx}
\usepackage{graphicx}
\usepackage[square,sort,comma,numbers]{natbib}
\usepackage{url}
\usepackage{enumitem}
\usepackage{booktabs}

\usepackage{caption} % To make fonts on figure smaller
\captionsetup[figure]{font=small}
\captionsetup[table]{font=small}


%opening
\title{Módulo 2: Estadística y probabilidad con python \newline Probabilidad}
\author{David R. Montalván Hernández}
\date{}

%=========Define los ambientes a utilizar =======%
%Define estilo para dar un salto de línea en el encabezado
%del 'teorema'
\newtheoremstyle{break}
{2ex} %above space
{2ex} %below space
{} %Body font)
{} %indent amount
{\bfseries} %head font
{:} %post head puncuation
{\newline} %post head space
{}

\theoremstyle{break}
%Definición
\newtheorem{definicion}{Definición}[chapter]

%Teorema
\newtheorem{teorema}{Teorema}[chapter]
\newtheorem*{demostracion}{Demostración}

%Proposición
\newtheorem{proposicion}{Proposición}[chapter]

%Notas importantes
\newtheorem{nota}{Nota}[chapter]

%Ejercicios
\newtheorem{ejercicio}{Ejercicio}[chapter]

%Ejemplos
\newtheorem{ejemplo}{Ejemplo}[chapter]

%Algoritmo (Utiliza el ambiente tabbing)
\theoremstyle{break}
\newtheorem{algoritmo}{Algoritmo}[chapter]
%=================================================%

%=================Macros===================%
\newcommand{\mbb}[1]{$\mathbb{#1}$}
\newcommand{\matdim}[2]{$#1 \times #2$}

\begin{document}
\sloppy %Para justificar correctamente (tiene que ver con \usepackage[none]{hyphenat})
\pagenumbering{Roman}
\maketitle
\renewcommand{\contentsname}{Contenido}
\tableofcontents
\renewcommand{\listfigurename}{Lista de imágenes}
\listoffigures
\renewcommand{\listtablename}{Lista de tablas}
\renewcommand\tablename{Tabla}
\renewcommand{\bibname}{Referencias}
\renewcommand{\figurename}{Figura}
\renewcommand{\chaptername}{Capítulo}
\listoftables

\chapter{Conceptos básicos de probabilidad}
\label{capitulo:conceptos basicos}

\section{Espacios y medidas de probabilidad}
\label{seccion:espacios y medidas de probabilidad}

\subsection{Espacios de probabilidad}
\label{seccion:Espacios de probabilidad}
%Espacio muestral y ejemplos (prob for ML pag 1)
%Definición de evento (prob for ML pag 2)
%Motivación y definición de sigma álgebra (Hoel pag 6)

\subsection{Medidas de probabilidad}
%Definición de probabilidad frecuentista (ML a Bayesian pag 11)
%Definición de medida de probabilidad (Hoel pag 8)
%Propiedades de una medida de probabildiad (Hoel 10)
%Fórmula de inclusión y exclusión (prob for ML pag 4)

\subsection{Probabilidad condicional e independencia}
%Definición de probabilidad condicional
%teorema 1.4 (prob for ML pag 5)
%Eventos mutuamente independientes (caso dos eventos y general Hoel pag 19)
%Ejemplo 8 Hoel pag 20

\chapter{Variables aleatorias}
\label{capitulo:variables aleatorias}
%Definición formal de una variable aleatoria
%Definición de una función de densidad
%Definición de una función de distribución
%Ejemplos de distribuciones
%Funciones de una variable aleatoria
%Independencia entre variables aleatorias
%Momentos de una variable aleatoria
%Fórmula de cambio de variable 
%Desigualdades famosas (prob for ML pag 19)
%Fórmula del jacobiano
%Funciones generadoras
%Distribuciones relacionadas a la distribución normal
%Ley de los grandes número y teorema del límite central
%Distribuciones multivariadas
%Distribuciones condicionales
%Ejemplo curse of dimensionality (prob for ML pag 131)
%Ejercicio esperanza condicional minimiza error cuadrático medio

\section{Caso discreto}
\label{seccion:variables discretas}

\section{Caso continuo}
\label{seccion:variables continuas}



\end{document}
